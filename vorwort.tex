
%\chapter*{Vorwort}
\addcontentsline{toc}{chapter}{Vorwort}
\thispagestyle{plain}
\lettrine[lines=3,loversize=0.2,slope=-15,lhang=0.2]{V}{or} euch liegt das neue Erstsemester-Infoheft 2017, in dem wir euch aktuelle Informationen zu euren Studiengängen, Modulen, Nebenfächern und vielem mehr bieten. Dieses Info-Heft eurer Fachschaft soll euch nicht nur im ersten Semester weiterhelfen, sondern auch während des Studiums immer wieder eine Nachschlagemöglichkeit bieten. Manche Dinge können sich natürlich mit der Zeit ändern, daher empfiehlt sich regelmäßig der Blick auf die Homepages der Institute und unserer Fachschaften. Bitte vergesst auch nicht, dass die Erstellung eines solchen Info-Heftes, besonders mit so vielen fleißig mithelfenden Menschen und so vielen beteiligten Studiengängen, immer eine Menge Arbeit bedeutet. Es ist wohl nicht zu vermeiden, dass sich der ein oder andere Fehler einschleicht oder eine Formatierung nicht optimal sitzt. Wir hoffen, ihr könnt dennoch etwas mit diesem Heft anfangen und wünschen euch viel Spaß beim Lesen!

Beim Personalbestand gibt es Neuerungen:
%, insbesondere im Bereich der Landschaftsökologie.
Wir begrüßen Prof. Dr. Samuel Mössner, der als Nachfolger von Prof. Dr. Ulrike Grabski Kieron die Leitung der Arbeitsgruppe Orts-, Regional-und Landesentwicklung / Raumplanung am Institut für Geographie übernommen hat. 

Noch ein letzter Absatz in eigener Sache: Um auch weiterhin erfolgreich zu arbeiten, benötigen wir eure Unterstützung. Wer wir sind und was wir eigentlich machen, findet ihr auf den folgenden Seiten oder könnt es auf unserer Homepage nachlesen. Wir würden uns freuen, wenn sich einige von euch angesprochen fühlen und Lust haben, einfach mal vorbeizukommmen und reinzuschnuppern. Wir sind ein lustiger Haufen, beißen nicht und spätestens bei einem gemütlichen Bierchen werdet ihr feststellen, dass Fachschaft nicht nur Arbeit bedeutet! Also viel Spaß beim Lesen und vor allem bei eurem Studium!
\bigskip
\newline
Eure Fachschaften GeoLök \& Geoinformatik
