\chapter{Bachelor of Science (B.Sc.) Geographie}
\lohead{\footnotesize{\textbf{Geographie} - Landschaftsökologie - Geoinformatik}}
\rehead{\footnotesize{\textbf{Geographie} - Landschaftsökologie - Geoinformatik}}

\section{Allgemeines}
Aus den alten Studienrichtungen „Physische Geographie“ und "`Sozialgeographie"' wurden 1992, im Zuge der Gründung des Institutes für Landschaftsökologie (ILÖK), flugs zwei verschiedene Studiengänge gemacht: Dipl. Geogr. und Dipl. LÖK. Später kam dann noch die Geoinformatik als eigenständiges Studium hinzu. Im Zuge der Europäisierung der Hochschulen (auch bekannt unter dem Namen Bologna-Prozess) wurden die Studiengänge auf das Bachelor/Master-System umgestellt und erfolgreich akkreditiert. Ihr seid mittlerweile der zwölfte Jahrgang des B.Sc. Geographie.

Planmäßig wird dieses Studium in sechs Semestern absolviert, danach könnt ihr entweder direkt ins Arbeitsleben eintauchen oder noch einen Master draufsetzen. Im Laufe dieses Studiengangs sollt ihr eine umfassende Grundbildung in Physio- und Anthropogeographie, mit einem deutlichen Übergewicht in letzterem, erhalten. Zudem sollen fundierte Kenntnisse in der wissenschaftlichen Methodik der Geographie vermittelt werden. Einige Module des Studiengangs fallen dabei in den Zuständigkeitsbereich anderer Institute. So ist zum Beispiel das ILÖK für nahezu alle Veranstaltungen im Bereich der Physischen Geographie zuständig. Ebenso werdet ihr einige Kurse am Institut für Geoinformatik (IfGI) belegen. Euer Nebenfach wählt ihr aus einer vorgegebenen Auswahl an Fächern aus, im Bereich der General Studies steht euch ein umfangreiches Kursangebot zur Verfügung, das individuell zusammengestellt werden kann.

Soweit die Theorie. Wie euer Studium konkret aussieht, wird durch die Prüfungsordnung festgelegt. Diese erhaltet ihr auf der Homepage des Instituts für Geographie (IfG). Wir haben in diesem Heft eine Übersicht des Studienverlaufsplans und eine Kurzbeschreibung der einzelnen Module eingefügt, jedoch ersetzt diese \textbf{IN KEINEM FALLE} die Prüfungsordnung. Wenn ihr hierzu Fragen habt, wendet euch an die zuständigen Dozenten bzw. Professoren oder fragt einfach uns – eure Fachschaft. Nachdem ihr euch dann irgendwann durch die Vorschriften und Studienordnungen gefuchst habt, stellt sich vielleicht auch irgendwann mal die Frage: Was macht eigentlich einen Geographen oder eine Geographin aus? Es ist gar nicht so einfach, diese Frage kurz und prägnant zu beantworten, aber versuchen wollen wir es trotzdem einmal.

Die Chance der Geographie liegt zum einen in ihrem breiten Themenfeld (von Natur- über Sozial- bis hin zu Geisteswissenschaften), zum anderen in ihrem variablen Maßstab (von der einzelnen Immobilie bis zum Globus samt Atmosphäre). GeographInnen werden in ihrem Studium mit derart unterschiedlichen Fächern und Fragestellungen konfrontiert, wie es wohl in keinem anderen Fach vorkommt. Natürlich, das behaupten viele Fächer von sich, aber fragt die doch dann einfach mal, ob sie an einem Tag Chemie (in der Bodenkunde), Marketing-Konzepte (in der Geographie des Einzelhandels) und Geschichte (in der Regionalen Geographie) behandeln. Diese Vielfalt erzeugt, wenn man es zulässt, ein breites, aber tendenziell etwas oberflächliches Wissen. Zudem wird vor allem die Fähigkeit geschult, verschiedene Ansätze, Fächer und Denkschulen miteinander zu verknüpfen, um bestimmte Probleme zu lösen.

Dieses vernetzte Denken ist die große Stärke der GeographInnen, die uns von Absolventen anderer Fächer unterscheidet. Es wird immer Menschen geben, die in gewissen Teilbereichen eine größere Kompetenz haben als wir. Es gilt aber, unsere Fähigkeiten mit allem Nachdruck und Selbstbewusstsein zu verkaufen, um nicht eines Tages bei Günther Jauch auf dem Stuhl zu sitzen mit dem Gedanken: "`Bitte keine Frage nach der Hauptstadt von Samoa!"'(Na, wisst ihr sie doch?). Oder buntfarbene Kartierungen als Picassos spätkubistische Phase auszuzeichnen, um der/dem Liebsten längere Erklärungen zu ersparen. In diesem Sinne: Lasst den Globus rotieren!

\section*{Was ist Geographie?}
Das ist wohl eine der schwersten Fragen, auf die jeder Geographiestudent vor oder während seines Studiums eine Antwort sucht. Einsteigen wollen wir bei der Suche nach Antworten mit einer Definition des Deutschen Verbandes für Angewandte Geographie (DVAG): „Geographie ist die Wissenschaft von den räumlichen Unterschieden und dynamischen Systemen“. Das ist erst einmal ein wenig abstrakt.

Man kann es sich vielleicht so vorstellen: Wenn man einen laufenden Film anhält und sich nur das Standbild betrachtet, kann man damit meistens nicht viel anfangen. Man versteht es erst, wenn man die Handlung und die Rollen der Akteure in ihrem Zusammenspiel kennt. Die Geographie forscht nach dem Zusammenspiel der Akteure im Film. „Wie gestalten wir unseren Lebensraum am besten?“ Unter Geographie versteht man also eine Vielzahl von Einzelaspekten, die alle Anteil am Gesamtsystem haben. Die meisten anderen Fachrichtungen beschränken sich auf einen Mosaikstein, während die Geographie versucht, viele dieser Mosaiksteine zusammenzulegen.

Dieses Kunststück schafft sie, wenn überhaupt, nur dadurch, dass sie sich in zwei recht unterschiedliche Bereiche teilt: die Physische Geographie (Natur und Umwelt) und die Anthropogeographie (Mensch und Umwelt). In der Geographie finden sowohl naturwissenschaftliche als auch sozialwissenschaftliche Methoden ihre Anwendung. Zusammenfassend kann also gesagt werden, dass die Geographie fächerübergreifend arbeitet und auch viele Gebiete beinhaltet, die auch von anderen wissenschaftlichen Disziplinen bearbeitet werden.

Eine noch aufschlussreichere Antwort auf die Frage \enquote{Was ist Geographie?} kann man vielleicht bekommen, wenn man aufzeigt, womit sich Geographiestudenten während ihres Studiums alles so beschäftigen.

\section*{Start in Münster}
Ihr möchtet jetzt wahrscheinlich wissen, was euch in der nächsten Zeit erwartet, wie euer Studienplan aussieht etc. Viele dieser Fragen werden während der Einführungswoche (Ersti-Woche) geklärt, daher solltet ihr nach Möglichkeit daran teilnehmen. Dort stellen sich die Dozenten und die Fachschaft vor und versuchen euch einen Einblick in das Studium zu verschaffen. Außerdem gibt es immer den einen oder anderen Kneipenabend und anschließend auch noch das sagenumwobene Ersti-Wochenende, um seine Kommilitonen ein wenig kennen zu lernen. Wir von der Fachschaft helfen euch, wo wir können, denn wir wissen wie es ist, als Erstsemester ein Studium in einer (für viele) neuen Stadt anzufangen.

Im ersten Semester steht in erster Linie die vierstündige Vorlesung \enquote{Einführung in die Humangeographie} an. Dazu gibt es noch die Ringvorlesung \enquote{Physische Geographie I} am ILÖK, sowie weitere Veranstaltungen. Vielleicht kommt euch euer Stundenplan etwas dürftig vor, aber Uni ist eben doch nicht dasselbe wie Schule, daher plant vor allem für \enquote{Einführung in die Humangeographie} eine Menge Zeit zum Vor- und Nachbereiten ein, da eine Menge Lesestoff anliegt und die Klausur am Ende des Semesters sehr lernaufwendig ist. Am besten ihr bleibt von Anfang des Semesters an am Ball, damit nicht irgendwann der Berg immer größer und die Zeit immer knapper wird. Lasst euch aber nicht von einer allgemeinen Panik-Mache mitreißen. Wenn man vernünftig für die Klausur lernt, dann besteht man sie auch!

\section{Studienverlaufsplan}
Der B.Sc. Geographie ist, wie bereits erwähnt, in sechs Semester unterteilt. Einige der Module erstrecken sich jedoch über zwei Semester, daher kann man eher von einer Einteilung in drei Jahre sprechen. Die ersten beiden Semester dienen dem Erlernen der Grundlagen der Geographie. Das zweite Studienjahr dient dem Aufbau des bereits erlangten Wissens, während die letzten beiden Semester dieses vertiefen sollen.

In diesen sechs Semestern müssen insgesamt 16 Module abgeschlossenen werden. Auf der folgenden Seite findet ihr den/einen Verlaufsplan eures Studiums. Wir empfehlen dringend, das erste Jahr wie im Verlaufsplan vorgegeben zu absolvieren, da die hier zu absolvierenden Module Voraussetzung für folgende Module sind. In den weiteren beiden Jahren können Kurse zum Teil auch früher oder später als im Verlaufsplan vorgegeben absolviert werden. Ihr solltet dabei jedoch immer im Auge behalten, dass ihr bestimmte Module benötigt, um andere belegen zu können. Die Zeiteinteilung des Nebenfaches ist unabhängig von den übrigen Modulen und kann sehr unterschiedlich ausfallen. Die Zeiteinteilung der General Studies könnt ihr komplett frei bestimmen.

Generell gilt: Der Studienaufbau ist stärker vorgegeben als dies noch beim Diplom der Fall war, allerdings gibt es hier und da auch Möglichkeiten etwas nach vorn oder hinten zu schieben. Letzteres kann aber auch schnell mal in die Hose gehen, so dass ein oder zwei Semester dran gehängt werden müssen.

%------------------------GRAPHIK FOLGT NOCH-----------------------------------------
%\newpage
%\includegraphics[angle=90, scale=0.5]{modulGEO}
%\newpage
%------------------------GRAPHIK FOLGT NOCH-----------------------------------------

\section{Modulbeschreibung}
\begin{enumerate}
 \item \textbf{Modul 1 ``Humangeographie 1a''}\\ besteht aus:
  \begin{enumerate}
   \item Vorlesung: Einführung in die Humangeographie
   \item Seminar: Humangeographie
   \item Tagesexkursion
  \end{enumerate}
  Notenvergabe:
  \begin{enumerate}
   \item Klausur \textbf{SEHR LERNINTENSIV!!!}
   \item Präsentation oder Hausarbeit
   \item Exkursionsbericht
  \end{enumerate}    % [] bewirkt, dass die nummerierung fehlt
  \item[] Note = a) 60\% der Note, b) 40\% der Note

 \item \textbf{Modul 2 ``Humangeographie 1b''} \\ besteht aus:
  \begin{enumerate}
   \item Übung: Einführung in das Studium der Geographie
   \item Seminar: Humangeographie
   \item Tagesexkursion
  \end{enumerate}
  Notenvergabe:
  \begin{enumerate}
   \item Präsentation
   \item Präsentationen oder Hausarbeit
   \item Exkursionsbericht
  \end{enumerate}
  \item[] Note = b) 100\% der Note

 \item \textbf{Modul 3 ``Physische Geographie I''}  \\ besteht aus:
  \begin{enumerate}
   \item Vorlesung: Physische Geographie
   \item Übung: Physische Geographie
  \end{enumerate}
  Notenvergabe:
  \begin{enumerate}
   \item Klausur
   \item Exkursionsprotokolle
  \end{enumerate}
  \item[] Note = a) 60\% der Note, b) 40\% der Note

 \item \textbf{Modul 4 ``Geographische Erhebungs- und Analysetechniken''}  \\ besteht aus:
  \begin{enumerate}
   \item Seminar: Kartographie und Karteninterpretation
   \item Seminar: Methoden der empirischen Humangeographie
   \item Übung: E-Learning"=Einheit zu „Kartographie und Karteninterpretation“
   \item Übung: E-Learning"=Einheit zu „Methoden der empirischen Humangeographie“
  \end{enumerate}
  Notenvergabe:
   \begin{enumerate}
    \item[] a) Übungsaufgaben; kartographische Arbeit
    \item[] b) Übungsaufgaben; Klausur
  \end{enumerate}
  \item[] Note = a) und b) jeweils 50\% der Note

 \item \textbf{Modul 5 ``Einführung in die Raumplanung''}  \\ besteht aus:
  \begin{enumerate}
   \item Vorlesung: Grundlagen der Raumplanung
   \item Seminar: Einführung in die räumliche Planung
   \item Tagesexkursion
  \end{enumerate}
  Notenvergabe:
  \begin{enumerate}
   \item Klausur
   \item Präsentation (inkl. schriftl. Ausarbeitung) und Planspiel
   \item Exkursionsprotokoll
  \end{enumerate}
  \item[] Note = a) 45\% b) 55\%

 \item \textbf{Modul 6a ``Geoinformatik 1a: Grundlagen''}  \\ besteht aus:
  \begin{enumerate}
   \item Vorlesung: Einführung in die Geoinformatik
   \item Übung: Einführung in die Geoinformatik
  \end{enumerate}
  Notenvergabe:
  \begin{enumerate}
   \item Klausur
   \item Übungsaufgaben
  \end{enumerate}
  \item[] Note = a) 100\%

 \item \textbf{Modul 6b ``Geoinformatik 1b: GIS Anwendungen''}  \\ besteht aus:
  \begin{enumerate}
   \item Übung: GIS-Grundkurs
   \item Übung: Angewandte Kartographie
  \end{enumerate}
  Notenvergabe:
  \begin{enumerate}
   \item Übungsaufgaben
   \item Thematische Karte
  \end{enumerate}
  \item[] Note = b) 100\%
    
 \item \textbf{Modul 7 ``Geoinformatik 2: Geostatistik''}  \\ besteht aus:
  \begin{enumerate}
   \item Vorlesung: Einführung in die Geostatistik
   \item Übung: Einführung in die Geostatistik
  \end{enumerate}
  Notenvergabe:
  \begin{enumerate}
   \item Klausur
   \item Übungsaufgaben
  \end{enumerate}
  \item[] Note = a) 100\%

 \item \textbf{Modul 8 ``Ökologische Planung''}  \\ besteht aus:
  \begin{enumerate}
   \item Vorlesung: Grundlagen der ökologischen Planung
   \item Übung: Grundlagen der ökologischen Planung
  \end{enumerate}
  Notenvergabe:
  \begin{enumerate}
   \item Klausur (oder mündl. Prüfung)
   \item Umweltbericht oder vergleichbares (Gruppenarbeit)
  \end{enumerate}
   \item[] Note = a) 100\%

 \item \textbf{Modul 9 ``Angewandte Geographie''}  \\ besteht aus:
  \begin{enumerate}
   \item Vorlesung: Angewandte Geographie
   \item Seminar 1: Angewandte Geographie
   \item Seminar 2: Angewandte Geographie
  \end{enumerate}
  Notenvergabe:
  \begin{enumerate}
   \item[] b) \& c) Präsentation
   \item[] b) oder c) Modul-Hausarbeit
  \end{enumerate}
  \item[] Note = b) oder c) 100\%

 \item \textbf{Modul 10 ``Geographie und Praxis''}  \\ besteht aus:
  \begin{enumerate}
   \item Übung: Berufsfelder der Geographie
   \item Seminar: Praktikumskolloquium (hören und Praktikum vorstellen)
   \item Praktikum (mindestens 4 Wochen)
  \end{enumerate}
  Notenvergabe:
  \begin{enumerate}
   \item[] b) Poster-Präsentation oder Praktikumsbericht
  \end{enumerate}
  \item[] Note = b) 100\%



 \item \textbf{Modul 11 ``Projektbezogenes Geländeseminar''}  \\ besteht aus:
  \begin{enumerate}
   \item Seminar: Projektseminar
   \item Projektbericht
  \end{enumerate}
  Notenvergabe:
  \begin{enumerate}
    \item[] b) Projektbericht und Präsentation
   \end{enumerate}
   \item[] Note = b) 100\%

 \item \textbf{Modul 12 ``Regionale Geographie''}  \\ besteht aus:
  \begin{enumerate}
   \item Vorlesung: Regionale Geographie
   \item Seminar: Regionale Geographie 1
   \item Seminar: Regionale Geographie 2
   \item Exkursion (6 Tage)
  \end{enumerate}
  Notenvergabe:
  \begin{enumerate}
   \item[] b) Präsentation
   \item[] c) Präsentation
   \item[] d) Präsentation und Exkursionsprotokoll
  \end{enumerate}
  \item[]  Note = d) 100\% 

 \item \textbf{Modul 13 ``Humangeographie 2''}  \\ besteht aus:
  \begin{enumerate}
   \item Vorlesung: Humangeographie 2
   \item Seminar: Humangeographie 2a
   \item Seminar: Humangeographie 2b
  \end{enumerate}
  Notenvergabe:
  \begin{enumerate}
   \item[] b) und c) Präsentation oder Hausarbeit
   \item[] Mündliche Modulabschlussprüfung (45 Min.)
  \end{enumerate}
  \item[] Note = Mündliche Modulabschlussprüfung 100\%

 \item \textbf{Modul 14 ``Allgemeine Studien''}  \\ besteht aus:
  \begin{enumerate}
   \item Veranstaltung aus dem Angebot der WWU
  \end{enumerate}
  Notenvergabe:
  \begin{enumerate}
   \item Abhängig von der jeweiligen Veranstaltung
  \end{enumerate}

 \item \textbf{Modul 15 ``Wahlbereich/Nebenfächer''}  \\
  Notenvergabe:
   \begin{enumerate}
    \item[] Abhängig von der jeweiligen Veranstaltung
   \end{enumerate}
 
 \item \textbf{Modul 16 ``Bachelorarbeit''}  \\ besteht aus:
  \begin{enumerate}
   \item Anfertigung eurer Bachelorarbeit 
   \item Umfang: 8.000 – 12.000 Worte (entspricht etwa 35-55 Seiten)
   \item Themenabsprache mit Dozenten
  \end{enumerate}




\end{enumerate}

\newpage

Mindestens ein außeruniversitäres Praktikum musst du im Laufe deines Studiums auf alle Fälle absolvieren. Dies soll laut Studienordnung mindestens 4 Wochen dauern. Wir empfehlen dir aber, das Praktikum länger durchzuführen. Wo und wann du dich um Stellen bewirbst ist dir überlassen. Es sei in diesem Rahmen darauf hingewiesen, dass das Praktikum einen Einblick in ein späteres Berufsleben geben soll. Du solltest diese Gelegenheit nutzen um dir erstens einen Überblick über das weite Spektrum der Tätigkeitsfelder von B.Sc./M.Sc Geographen/Innen zu verschaffen und zweitens um festzustellen, ob dir eine spätere (lebenslängliche?) Tätigkeit in einem Bereich überhaupt zusagt.

Der beste Zeitpunkt für ein Praktikum ist vermutlich nach dem dritten oder vierten Semester in den Semesterferien. Dann hat man schon ein bisschen Ahnung, wovon man eigentlich redet, kann aber auf der anderen Seite auch sein Studium noch etwas beeinflussen. Es empfiehlt sich jedoch, mehr als nur dieses eine Pflichtpraktikum zu absolvieren, da praktische Erfahrung zum einen euch selber mehr zu Gute kommt als vielleicht eine weitere absolvierte theoretische Veranstaltung, zum anderen wird diese Erfahrung durch spätere Arbeitgeber gern gesehen. Zudem können durch vernünftig durchgeführte Praktika Kontakte geknüpft werden, welche für den Berufseinstieg sehr hilfreich sind.

\section*{Wahlbereich/Nebenfächer}
Im so genannten Wahlbereich müsst ihr 30 Credit Points in einem bzw. mehreren Nebenfächern belegen. Derzeit können öffentliches Recht, VWL, Geoinformatik, Politikwissenschaften, Geowissenschaften, Niederlande-Studien und Landschaftsökologie als Nebenfach belegt werden. Die ersten drei genannten Fächer verlangen, dass die gesamten 30 CP in ihrem Fach belegt werden. Bei den letzteren vier können diese aus Modulen à 10 CP zusammengesetzt werden, sodass z.B. 10 CP in Geowissenschaften und 20 CP in Landschaftsökologie belegt werden können. Die Wahl des Nebenfachs erfolgt erst nach Vorlesungsbeginn. In der ersten Woche könnt ihr ruhig mehrere Einführungsvorlesungen besuchen und dann schauen, was euch am Besten gefällt. Denkt bei der Auswahl am Ende zum einen daran, was euch am meisten interessiert, zum anderen aber auch, was davon zu einem etwaigen späteren Tätigkeitsfeld passen könnte. Ausführlichere Informationen zu eurem Studiengang inklusive der Nebenfächer findet ihr unter: \url{www.uni-muenster.de/Geographie/studium/studiengang}

\section*{Das Berufsfeld der Geographie}
Um es vorwegzunehmen: es gibt kein eindeutiges Berufsfeld für Absolventen der Geographie. Die Möglichkeiten, im Berufsleben unterzukommen, sind äußerst vielfältig (das ist schön), das Arbeitsplatzangebot ist gering (das ist schade) und die Konkurrenz mit anderen qualifizierten Menschen aus anderen qualifizierten Bereichen ist nicht zu unterschätzen (das ist normal). Soll heißen: es ist alles und nichts möglich, vom Flughafendirektor (z.B. MS-OS) über die im Stadtplanungsamt tätige Verkehrs- oder Raumplanerin oder als Spezialistin beim Katastrophenmanagement bis hin zum Journalist werden viele Spektren der Gesellschaft durch Geos frequentiert. Auch eine Karriere als selbständige/r GeographIn ist denkbar und hat schon zu durchaus erfolgreichen Ergebnissen geführt. Daher soll hier auch auf diese o. ä. Fragen keine Antwort gegeben werden, schließlich soll die Antwort das Ergebnis deines Studiums sein (O.K..., das klingt ja doch ein wenig pathetisch...). Im Laufe deines Studiums wirst du wahrscheinlich genügend ökologische Nischen finden, in denen du dir dein späteres, werktätiges Leben vorstellen kannst, womit wir zum Berufsmarkt kommen: Als ich zu Beginn meines Studiums jemandem von meinem Fach erzählte, hörte ich: „Wieso? Wir wissen doch inzwischen, dass die Erde eine Kugel ist?!“.

Dass Geographen keine Entdeckungen mehr tätigen, ist inzwischen bekannt. Seit den 60ern hat sich dafür die Raum- und Umweltplanung als relativ großer Arbeitsmarkt für Geographen etabliert. Da bei der öffentlichen Hand allerdings immer mehr eingespart wird und sich andererseits mittlerweile auch hierfür Spezialisten (Studiengang Raumplanung) entwickelt haben, ist es nicht mehr so leicht „auf dem Amt“ eine Stelle zu bekommen – und sicherlich auch nicht für jedermann erstrebenswert. Gerade in diesem Arbeitsfeld, der planerischen Gestaltung unserer Lebensumwelt, hat sich aber dennoch vieles getan. Es bilden sich immer mehr regionale oder lokale Netzwerke, zum Beispiel durch den Zusammenschluss von Kreisen oder Städten zu einer Region, bei lokalen Einzelhandelsgemeinschaften oder auch bei der interkommunalen Zusammenarbeit über Staatsgrenzen hinaus. In diesen informellen Netzwerken entstehen oft Arbeitsmöglichkeiten für GeographInnen, ob als Angestellte oder in selbstständigen Agenturen. Weitere Arbeitsfelder in diesem Umfeld sind der Tourismusbereich, die Wirtschaftsförderung von Kommunen und Kreisen oder das Stadtmarketing.

Seit einigen Jahren nimmt auch die Bedeutung der Verarbeitung von geographischen Daten stetig zu. Hier reicht das Spektrum von der klassischen Kartenerstellung über die Entwicklung und Wartung von Navigationssystemen und Routenplanern sowie die Erfassung von Vegetation und Bodenqualität bis hin zur kompletten Erstellung von geoinformatischen Datenbanken und deren Verwaltung. Darüber hinaus sind Geographen aber auch in Entwicklungszusammenarbeit und Katastrophenhilfe, Erwachsenenbildung, Unternehmensberatung, öffentlicher Verwaltung, Lobbyarbeit, Politik, Presse und Medien, Immobilienwirtschaft und vielen anderen Bereichen anzutreffen.

Hier konkurriert man natürlich immer mit den „Spezialisten“, also BWLern, Pädagogen, Politikwissenschaftler, Ingenieure, Juristen usw. Gerade hier ist es also wichtig, das breite Wissen und die Fähigkeit zum interdisziplinären Denken offensiv zu vertreten. Also, ihr seht schon: Es gibt keine vorgezeichnete Karriere für Geographen, wie etwa bei Medizinern oder Lehrern. Man sollte deshalb aber nicht in Panik oder Fatalismus verfallen. Macht euch einfach ein paar Gedanken, in welche Richtung ihr gehen wollt. Informiert euch, was es da für Firmen, Organisationen oder Behörden gibt. Macht vielleicht auch ein oder zwei Praktika mehr, als es in eurer Studienordnung mindestens gefordert wird, es lohnt sich in jedem Fall! Und merkt euch immer: Wir Geographen sind die letzten Spezialisten fürs Allgemeine! Weitere Informationen über mögliche Berufsfelder findet ihr auf der Homepage des Deutschen Verbands für angewandte Geographie (DVAG): \url{http://www.geographie.de/dvag/}

\section*{Exkurs: Geheimnisvoller Geograph}
Angenommen, bei einem Sektempfang gibt sich einer der Gäste als Quantenphysiker zu erkennen: Wer würde da nicht vor Ehrfurcht erstarren und sofort das Thema wechseln, um sich keine Blöße zu geben? Was aber, wenn die Partybekanntschaft kein Quantenphysiker ist, sondern ein Geograph? Das wäre weniger problematisch, schließlich hatte jeder mal Erdkunde in der Schule, musste drei Nebenflüsse der Weser aufsagen, den höchsten Berg von Frankreich nennen und hatte zu lernen, wo die Apfelsinen wachsen.

Das einzige, was einem jetzt noch schleierhaft sein kann, wäre, wie jemand mit solchem Wissen Geld verdient. Wollte er das erklären, müsste der neue Bekannte die Geographie beschreiben, müsste von Biotoppflege, Erosionsforschung und Bodenverdichtung berichten, müsste von Satellitenbildern erzählen und von Umweltgutachten, von Raumplanung und Dorferneuerung, von Stadtmarketing und politischer Beratung, von Standortgutachten und Moderation.

Da er das schon so oft hat herbeten müssen, könnte er unwirsch behaupten, Geographen seien so etwas Ähnliches wie Geologen. Zwar wissen längst nicht alle Menschen, dass Geologen Spezialisten für den Aufbau der Erde sind, für die Bildung von Gesteinen und die Lage von Bodenschätzen. Gerade deswegen stößt ihr Beruf in der Gesellschaft auf ähnlichen Respekt wie der des Quantenphysikers. Indem er sich als Geologe ausgibt, würde der Bekannte zudem eine Menge Zeit sparen, weil man ihn am Tag nach dem Sektempfang ohnedies dafür halten würde - ganz so, wie sein Frisör das tut oder seine alte Tante, auch wenn sie den Unterschied schon fünfmal erklärt bekam.

Das Ansehen der Geographen leidet darunter, dass so viele Menschen früher den Erdkundeunterricht genossen oder besser erlitten haben und meinen, die moderne Geographie sei dasselbe. Dummerweise gingen auch Personalchefs früher mal zur Schule. Wer ihnen als Stellenbewerber den Job etwas erleichtern möchte, gibt sich am besten gleich als Geologe aus. Als solcher wird er zwar auch abgelehnt, aber der Personalchef hätte wenigstens das gute Gefühl zu wissen, wen er da wieder nach Hause geschickt hat. Schuld am verschwommenen Berufsprofil ist, dass Geographen sich für alles zuständig fühlen, womit sie so falsch nicht liegen. Denn ihre Ausbildung streift außer Mikroelektronik und indoiranischer Linguistik so ziemlich alles, was Universitäten an Fächern zu bieten haben.

Böse Zungen behaupten, Geographie studiere nur, wer seit einem misslungenen Knallgas-Versuch am Gymnasium ein gebrochenes Verhältnis zur Naturwissenschaft habe, oder für die Betriebswirtschaft nicht in Frage kommt, weil vom Vater kein Geschäft zu übernehmen ist. Wenn sie unter sich sind, bezeichnen sich Geographen selber als Universaldilettanten. Das hält sie nicht davon ab, über andere Disziplinen zu spotten, wo man über unendlich dimensionale Hilbert-Räume promovieren kann, ohne zu wissen, wie man einen Dreisatz berechnet. Zu Fachidioten können aber selbst Geographen werden, die letzten Spezialisten fürs Ganze: Einer schrieb bestimmt eine Diplomarbeit über die Verteilung der Imbissbuden oder Zigarettenautomaten in einer Großstadt. Man sagt das nicht gerne. Aber auch das ist Geographie. 

(Die Zeit Nr. 20 / 14. Mai 1993 – Autor: Walter Schmidt - leicht verändert)

\section*{Schlusswort}
Wie ihr seht, ist euer Studium relativ stark durchgeplant. Daher ist es besonders wichtig, dass ihr euch den Ablauf verinnerlicht. Genau so wichtig ist es, dass ihr euch an Fristen und Termine haltet und diesbezüglich immer auf dem Laufenden seid. Durch verpennte Termine oder verpatzte Klausuren kann man schnell viel Zeit verlieren. Trotz der vielen Formalien hoffen wir, euch nicht zu sehr vom Studium abgeschreckt zu haben. Studieren ist, bis auf die Prüfungen und manchmal viel Arbeit mit Referaten, Hausarbeiten etc., doch eine ganz nette Angelegenheit. Ihr solltet euch auf jeden Fall im ersten Semester neben dem Arbeiten in der Uni genug Zeit nehmen, um Leute kennen zu lernen und Münsters nicht zu unterschätzendes Nachtleben zu erforschen. Wer ausgiebige Informationen über Parties und Locations haben möchte, kann sich auch hier gerne an die Fachschaft richten. Wir haben da auf jeden Fall genügend Erfahrungen. 
Die Fachschaft wünscht euch viel Spaß und wenn’s Probleme gibt, kommt einfach zu uns: wir werden versuchen euch mit Rat und Tat zur Seite zu stehen!
